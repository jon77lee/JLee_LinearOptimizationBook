
%  LaTeX_Template.tex // J. Lee
%
% ----------------------------------------------------------------
% AMS-LaTeX ************************************************ 
% **** -----------------------------------------------------------
\documentclass{amsart}
\usepackage{graphicx,amsmath,amsthm}
\usepackage{hyperref}
\usepackage{verbatim}
\usepackage[a4paper,text={16.5cm,25.2cm},centering]{geometry}
% ----------------------------------------------------------------
\vfuzz2pt % Don't report over-full v-boxes if over-edge is small
\hfuzz2pt % Don't report over-full h-boxes if over-edge is small
% THEOREMS -------------------------------------------------------
\newtheorem{thm}{Theorem}[section]
\newtheorem{cor}[thm]{Corollary}
\newtheorem{lem}[thm]{Lemma}
\newtheorem{prop}[thm]{Proposition}
\theoremstyle{definition}
\newtheorem{defn}[thm]{Definition}
\theoremstyle{remark}
\newtheorem{rem}[thm]{Remark}
\numberwithin{equation}{section}
% MATH -----------------------------------------------------------
\newcommand{\Real}{\mathbb R}
\newcommand{\eps}{\varepsilon}
\newcommand{\To}{\longrightarrow}
\newcommand{\BX}{\mathbf{B}(X)}
\newcommand{\A}{\mathcal{A}}
% ----------------------------------------------------------------
\begin{document}

\title{\LaTeX~ Template}

\date{\today}

\maketitle

 \href{mailto:youruniqname@umich.edu,yourfrienduniqname@umich.edu,anotheruniqname@umich.edu}
{Your Actual Name (youruniqname@umich.edu), 
Your Friend's Actual Name (yourfrienduniqname@umich.edu), 
Another Actual Name (anotheruniqname@umich.edu)}

\medskip

(this identifies your work and it \emph{greatly} help's me in returning homework to you by email
---- just plug in the appropriate replacements in the \LaTeX~ source; then when I click on the 
hyperlink above, my email program opens up starting a message to you)

\bigskip

% ----------------------------------------------------------------

This template can serve as a starting point for learning \LaTeX. Download MiKTeX from
{\tt miktex.org}
to get started. Look at the source file for this
document (in Section \ref{sec:appendix})
to see how to get all of the effects demonstrated.

\section{This is the first section where we make some lists}

It is easy to make enumerated lists:
\begin{enumerate}
\item This is the first item
\item Here is the second
\end{enumerate}

And even enumerated sublists:
\begin{enumerate}
\item This is the first item
\item Here is the second with a sublist
\begin{enumerate}
\item first sublist item
\item and here is the second
\end{enumerate}
\end{enumerate}

\section{Here is a second section where we typeset some math}

You can typeset math inline, like $\sum_{j=1}^n a_{ij} x_j$, by just enclosing the math in dollar signs.

But if you want to \emph{display} the math, then you do it like this:

\[
\sum_{j=1}^n a_{ij} x_j~ \forall~ i=1,\ldots,m.
\]

And here is a matrix:
\[
\left(
  \begin{array}{ccccc}
    1 & \pi & 2& \frac{1}{2} & \nu \\
    6.2 & r & 2 & 4 & 5 \\
    |y'| & \mathcal{R} & \mathbb{R} & \underbar{r} & \hat{R} \\
  \end{array}
\right)
\]

Here is an equation array, with the equal signs nicely aligned:
\begin{eqnarray}
  \sum_{j=1}^n x_j &=& 5 \label{E1} \\
    \sum_{j=1}^n y_j &=& 7 \label{E7} \\
  \sum_{j\in S} x_j &=& 29 \label{E4}
\end{eqnarray}

The equations are automatically numbered, like $x.y$, where
$x$ is the section number and $y$ is the $y$-th equation in section $x$.
By tagging the equations
with labels, we can refer to them later, like (\ref{E4}) and (\ref{E1}).

\begin{thm}\label{Favorite}
This is my favorite Theorem.
\end{thm}
\begin{proof}
Unfortunately, the space here does not allow for including my ingenious proof
of Theorem \ref{Favorite}.
\end{proof}

\section{Here is how I typset a standard-form linear-optimization problem}

\[
\tag{P}
\begin{array}{rrcl}
 \min & c'x  &      &   \\
      &  Ax  &   =  & b~; \\
      &   x  & \geq & \mathbf{0}~.
\end{array}
\]

Notice that in this example, there are 4 columns separated by 3 \&'s.
The 'rrcl' organizes justification within a column.
Of course, one can make more columns.

\section{Graphics}

This is how to include and refer to Figure \ref{nameoffigure} with pdfLaTeX.

\begin{figure}[h!!]
\includegraphics[width=0.5\textwidth]{yinyang.jpg}
\caption{Another duality}\label{nameoffigure}
\end{figure}

\section{The \LaTeX~ commands to produce this document}
\label{sec:appendix}

Look at the \LaTeX~ commands in this section to see how each of the elements 
of this document was produced. Also, this section serves to show
how text files (e.g., programs) can be included in a \LaTeX~ document verbatim.

\bigskip

\hrule

\small
\verbatiminput{LaTeX_Template.tex}
\normalsize

\hrule

\bigskip

% ----------------------------------------------------------------

\end{document}
% ----------------------------------------------------------------


